\documentclass[a4paper]{ctexart}    %文类(论文)
%目录层级
\setcounter{secnumdepth}{4}
\setcounter{tocdepth}{4}
% 宏包
\usepackage{ctex}              	%支持中文
\usepackage{amsmath}           	%数学公式
\usepackage{amssymb}
\usepackage{graphicx}          	%插图
\usepackage{listings}          	%插入代码
\usepackage{color, xcolor}     	%颜色
\usepackage{longtable, booktabs}	%三线表
\usepackage{pdfpages}			%导入pdf
\usepackage{titlesec}			%设置标题、子标题
\usepackage{subfigure}			%子图
\usepackage{tcolorbox}			%彩色盒子
\usepackage{multirow}			%不规则表格

%页边距调整
\usepackage[left=2.54cm, bottom=3.18cm, right=2.54cm, top=3.18cm]{geometry}
\usepackage{fancyhdr}    		%页面样式
\pagestyle{fancy}    			%fancy页面样式

%页眉页脚
\fancyhead[r]{\textbf{贤妃的数学笔记}}
\fancyfoot[r]{\textbf{cg系统数理统计作业}}
%标题
\title{\textbf{Statistics Note}}
\author{by xianfei}
\date{}

\begin{document}
	\maketitle
	\noindent
	\thispagestyle{empty}
	\begin{table}[htbp!]
		\centering
		\begin{tabular}{c}
			\Large Last math class, my younth ends! \\
			\\
			\Large 先看附录! \\ \\
			\huge 先看附录! \\ \\
			\Huge 先看附录! 
		\end{tabular}
	\end{table}
	\newpage
	\thispagestyle{empty}
	\tableofcontents
	\begin{figure}[htbp!]
		\centering
		\includegraphics{huaji1.png}
	\end{figure}
	\newpage
	\section{特征函数作业}
	\noindent
	\begin{tcolorbox}
		[colframe=blue!25,
		colback=blue!10,
		coltitle=blue!20!black,  
		fonttitle=\bfseries,
		adjusted title=Formula Or Theorem:
		]
		\begin{itemize}
			\item \textbf{定义 4.2.1(P193)} 设$ X $是一个随机变量,称
			\begin{equation*}
				\varphi(t) = E(e^{itX}), \qquad -\infty < t < +\infty
			\end{equation*}
			为$ X $的\textbf{特征函数}
			\item \textbf{性质4.2.3(P194)} 若$ Y = aX + b $,其中$ a $,$ b $是常数,则
			\begin{equation*}
				\varphi_{Y}(t) = e^{ibt}\varphi_{X} (at)
			\end{equation*}
			\item \textbf{性质4.2.4(P194)} 独立随机变量和的特征函数为每个随机变量的特征函数的积,即设$ X $与$ Y $相互独立,则
			\begin{equation*}
				\varphi_{X+Y}(t) = \varphi_{X}(t)\varphi_{Y}(t)
			\end{equation*}
			\item \textbf{性质4.2.5(P194)} 若$ E(X^{l}) $存在,则$ X $的特征函数$ \varphi(t) $可$ l $次求导,且对$ 1 \leq k \leq l $,有
			\begin{equation*}
				\varphi^{(k)}(0) = i^kE(X^k)
			\end{equation*}
			特别地
			\begin{equation*}
				E(X) = \frac{\varphi'(0)}{i} \quad , \quad Var(X) = -\varphi''(0) + (\varphi'(0))^2
			\end{equation*}
		\end{itemize}
	\end{tcolorbox}
	\noindent
	1.1 \\
	由定义4.2.1得
	\begin{equation*}
		\begin{split}
			\varphi (t) & = E(e^{itx_k}) = \sum\limits_{k=0}^{3} e^{itk}p_k\\
			& = 0.1\cdot e^{3it} + 0.2\cdot e^{2it} + 0.3\cdot e^{it} + 0.4
		\end{split}
	\end{equation*}
	\\
	1.2 \\
	由定义4.2.1得
	\begin{equation}
		\varphi (t) = E(e^{itx_k}) = \sum\limits_{k=1}^{\infty} e^{itk} \cdot p(1-p)^{k-1}
		= pe^{it}\sum\limits_{k=1}^{\infty}\left[e^{it}(1-p)\right]^{k-1}\tag{$*$}
	\end{equation}
	式$(*)$中,$\left|e^{it}(1-p)\right| \leq \left|e^{it}\right| \cdot \left|(1-p)\right| < 1$得
	\begin{equation*}
		\varphi (t) = \frac{pe^{it}}{1-e^{it}(1-p)}
	\end{equation*}
	即得
	\begin{equation*}
		\begin{split}
			\varphi^{'}(t) = \frac{ipe^{it}}{(1-e^{it}(1-p))^2}\qquad\qquad
			\varphi^{''}(t) = \frac{-pe^{it}(1+e^{it}(1-p))}{(1-e^{it}(1-p))^3}
		\end{split}
	\end{equation*}
	由性质4.2.5得数学期望、方差
	\begin{equation*}
		\begin{split}
			E(X) &= \frac{\varphi^{'}(0)}{i} = \frac{1}{p} \\
			Var(X) &= -\varphi^{''}(0) + (\varphi^{'}(0))^2 = \frac{1-p}{p^2}
		\end{split}
	\end{equation*}
	\section{大数定律与中心极限定理作业}
	\begin{tcolorbox}
		[colframe=blue!25,
		colback=blue!10,
		coltitle=blue!20!black,  
		fonttitle=\bfseries,
		adjusted title=Formula Or Theorem:
		]
		\begin{itemize}
			\item \textbf{定义4.3.1 (P206)} 设有一随机变量序列$ \{X_n\} $,如果它对任意的$ \varepsilon > 0 $,满足
			\begin{equation}
				\lim\limits_{n \to \infty}P\left(\bigg|\frac{1}{n}\sum\limits_{i=1}^{n}X_i-\frac{1}{n}\sum\limits_{i=1}^nE(X_i)\bigg| < \varepsilon\right) = 1\tag{4.3.1}
			\end{equation}
			形式,则称该随机变量序列$ \{X_n\} $服从\textbf{大数定律}
			\item \textbf{定理2.3.1(Chebyshev不等式 P80)} 设随机变量$ X $的数学期望和方差都存在,则对任意常数$ \varepsilon > 0 $,有
			\begin{equation*}
				P(|X - E(X)| \geq \varepsilon) \leq \frac{Var(X)}{\varepsilon^2}
			\end{equation*}
			或
			\begin{equation*}
				P(|X-E(X)| < \varepsilon) \geq 1-\frac{Var(X)}{\varepsilon^2}
			\end{equation*}
			\item \textbf{定理4.3.2(Chebyshev大数定律 P206)} 设$ \{X_n\} $为一列两两不相关的随机变量序列,若每个$ X_i $的方差存在,且有共同上界,即$ Var(X_i) \leq c\text{,} i=1, 2, \cdots $,则$ \{X_n\} $服从大数定律,即对任意的$ \varepsilon > 0 $,\textbf{式4.3.1}成立\textbf{(详见书)}
			\item \textbf{定理4.3.4(Khinchin大数定律 P207)} 设$ \{X_n\} $为一独立同分布的随机变量序列,若$ X_i $的数学期望存在,则$ \{X_n\} $服从大数定律,即对任意的$ \varepsilon > 0 $,\textbf{式4.3.1}成立\textbf{(详见书)}
		\end{itemize}
	\end{tcolorbox}
	\noindent
	2.1 \quad\textbf{D} \\
	由$X\sim P(2)$,所以$E(X)=2$,$Var(X)=2$\\
	由$Chebyshev$不等式
	\begin{equation*}
		\begin{split}
			&P\left(\left|X-E(X)\right|\geq \varepsilon \right) \leq \frac{Var(X)}{\varepsilon^2}\\
			&P\left(\left|X-E(X)\right|< \varepsilon\right) \geq 1 - \frac{Var(X)}{\varepsilon^2}
		\end{split}
	\end{equation*}
	令$\varepsilon = 2$即可\\
	\\
	2.2 \quad\textbf{D} \\
	先求$E(X+Y)$,$Var(X+Y)$
	\[E(X+Y) = E(X) + E(Y) = 0\]
	\begin{equation*}
		\begin{cases}
			Var(X+Y)=Var(X)+Var(Y)+2Cov(X,Y)\\
			Corr(X,Y)=\frac{Cov(X,Y)}{\sigma_X \sigma_Y}
		\end{cases}
	\end{equation*}
	解得$Var(X+Y)=3$,再由$Chebyshev$不等式即可得到下界\\
	\\
	2.3 \quad\textbf{D} \\
	由题意知$X_i \sim Exp(2)$,即得$E(X_i)=1/2$,$ E(X_i^2)=1/2 $\\
	${X_n}$是独立同分布,由$Khinchin$大数定理
	\begin{equation*}
		\frac1n\sum\limits_{i=1}^{n}X_i^2-(\frac1n\sum\limits_{i=1}^{n}X_i)^2 \xrightarrow{\quad P\quad} E(X_n^2) - (E(X_n))^2=\frac14
	\end{equation*}
	\\
	2.4 \quad\textbf{A} \\
	由$Khinchin$大数定理
	\begin{equation*}
		\frac1n \sum\limits_{i=1}^{n}X_i(X_i-1)=\frac1n\sum\limits_{i=1}^{n}X_i^2-\frac1n\sum\limits_{i=1}^{n}X_i \xrightarrow{\quad P \quad} E(X_n^2)-E(X_n) = 1
	\end{equation*}
	\begin{tcolorbox}
		[colframe=blue!25,				%边框颜色
		colback=blue!10,				%背景颜色
		coltitle=blue!20!black, 		%标题颜色 
		fonttitle=\bfseries,			%标题字体
		adjusted title=Formula Or Theorem:	%盒子标题
		]
		\begin{itemize}
			\item \textbf{定理4.4.1(Lindeberg-Levy中心极限定理 P212)} 设$ X_n $是独立同分布的随机变量序列,且$ E(X_i) = \mu $,$ Var(X_i)=\sigma^2 > 0 $存在,若记
			\begin{equation*}
				Y_n^* = \frac{X_1 + X_2 + \cdots + X_n - n\mu}{\sigma \sqrt{n}}
			\end{equation*}
			则对任意实数$ y $,有
			\begin{equation*}
				\lim\limits_{n \to \infty}P(Y_n^* \leq y) = \varPhi(y) = \frac{1}{\sqrt{2\pi}}\int_{-\infty}^{y}e^{-\frac{t^2}{2}}dt
			\end{equation*}
			\item \textbf{定理4.4.2(de Moivre-Laplace中心极限定理 P214)} 设$ n $重伯努利实验中,事件A在每次试验中出现的概率为$ p (0<p<1) $,记$ S_n $为$ n $次试验中事件A出现的次数,且记
			\begin{equation*}
				Y_n^*=\frac{S_n-np}{\sqrt{npq}}
			\end{equation*}
			则对任意实数y,有
			\begin{equation*}
				\lim\limits_{n \to \infty}P(Y_n^* \leq y) = \varPhi(y) = \frac{1}{\sqrt{2\pi}}\int_{-\infty}^{y}e^{-\frac{t^2}{2}}dt
			\end{equation*}
		\end{itemize}
	\end{tcolorbox}
	\noindent
	2.5 \quad\textbf{B} \\
	设最多可以装载$n$件产品,由题意
	\begin{equation*}
		P\left(\sum\limits_{i=1}^{n} X_i \leq 5000 \right) = 0.99 \tag{$*$}
	\end{equation*}
	利用$Lindeberg-Levy$中心极限定理得
	\begin{equation*}
		(*)=P\left(\frac{\sum\limits_{i=1}^{n}X_i-50n}{\sigma \sqrt{n}} \leq \frac{5000-50n}{\sigma \sqrt{n}}\right)\approx \Phi\left(\frac{5000-50n}{5 \sqrt{n}} \right)=0.99
	\end{equation*}
	解得$n_{max}=95$\\
	\\
	2.6 \quad\textbf{C} \\
	设第$i$次称重的重量为$X_i$,易知$\overline{X}_n=\left(\sum\limits_{i=1}^{n}X_i \right)/ n$\\
	由$Lindeberg-Levy$中心极限定理
	\begin{equation*}
		P\left(\left|\frac{\sum\limits_{i=1}^{n}X_i-na}{\sigma\sqrt{n}}\right| < \frac{0.1 n}{\sigma \sqrt{n}}\right)\approx 2\Phi\left(\frac{0.1n}{\sigma \sqrt{n}}\right)-1 \geq 0.95
	\end{equation*}
	解得$n_{min} = 16$\\
	\\
	2.7 \quad\textbf{B} \\
	由题意得$X_i \sim P(1)$,则$E(X_i)=1$,$Var(X_i)=1$\\
	\begin{equation*}
		\begin{split}
			&\lim\limits_{n \to \infty}P\left(\sum\limits_{i=1}^{n}X_i > n\right)=1-\lim\limits_{n \to \infty}P\left(\sum\limits_{i=1}^{n}X_i \leq n \right)\\
			&=1-\lim\limits_{n \to \infty}P\left(\frac{\sum\limits_{i=1}^{n}X_i-n}{\sigma\sqrt{n}} \leq 0\right)=1-\Phi(0)=0.5
		\end{split}
	\end{equation*}
	2.8 \quad\textbf{C}\\
	易得
	\begin{equation*}
		\begin{split}
			Var(X_i-Y_i)=Var(X_i)+Var(Y_i)=\frac{13}{3}\\
			E(X_i-Y_i)=E(X_i)-E(Y_i)=0
		\end{split}
	\end{equation*}
	由$Lindeberg-Levy$中心极限定理
	\begin{equation*}
		\begin{split}
			&\lim\limits_{n \to \infty} P\left(\sum\limits_{i=1}^{n}(X_i-Y_i) > 0\right)=1-\lim\limits_{n \to \infty} P\left(\sum\limits_{i=1}^{n}(X_i-Y_i) \leq 0\right)\\
			&=1-P\left(\frac{\sum\limits_{i=1}^{n}(X_i-Y_i)}{\sigma \sqrt{n}} \leq 0\right) = 1-\Phi(0) = 0.5
		\end{split}
	\end{equation*}
	\\
	2.9 \quad\textbf{C} \\
	由$deMoivre-Laplace$中心极限定理得
	\begin{equation*}
		\begin{split}
			P(14 < X < 30) &\approx \Phi\left(\frac{30-20+0.5}{4}\right)-\Phi\left(\frac{14-20-0.5}{4}\right)\\
			&=\Phi(2.63)+\Phi(1.63)-1\approx 0.944
		\end{split}
	\end{equation*}
	\textbf{未修正的结果为}
	\[P(14 < X < 30)=\Phi(2.5)+\Phi(1.5)-1\approx 0.927\] \\
	2.10 \\
	设每袋味精的质量为随机变量$X_i(i=1,..,200)$,记随机变量序列为$\{X_n\}$,即求
	\begin{equation*}
		P\left(\sum\limits_{i=1}^{200}X_i > 20500\right)\tag{$*$}
	\end{equation*}
	由$Lindeberg-Levy$中心极限定理得
	\[(*) = P\left(\frac{\sum\limits_{i=1}^{200}X_i-100\cdot 200}{10\cdot \sqrt{200}} > \frac{500}{100\cdot \sqrt{2}}\right)\approx 1-\Phi\left(\frac{5}{\sqrt{2}}\right)\approx 0.0002326\] \\
	2.11 \\
	设每次命中的环数为随机变量$X_i$,记随机变量序列为$\{X_n\}$,其中\\
	\[E(X_i)=10\times 0.8 + 9\times 0.1 + 8\times 0.05 + 7\times 0.02 + 6\times 0.03 = 9.62\]
	\[Var(X_i)=E(X_i^2)-(E(X_i))^2=0.82\text{,}\sigma=\sqrt{Var(X_i)}=0.91\]
	求100次射击命中环数在900环到930环之间的概率,即求
	\begin{equation*}
		P\left(900 \leq \sum\limits_{i=1}^{n}X_i \leq 930\right)\tag{$*$}
	\end{equation*}
	由$Lindeberg-Levy$中心极限定理得
	\begin{equation*}
		\begin{split}
			(*) &= P\left(\frac{900-100\times 9.62}{0.91 \times 10} \leq \frac{\sum\limits_{i=1}^{n}X_i-100\times 9.62}{0.91\times 10} \leq \frac{930-100\times 9.62}{0.91\times 10}\right)\\
			&\approx \Phi\left(\frac{930-100\times 9.62}{0.91\times 10}\right) - \Phi\left(\frac{900-100\times 9.62}{0.91 \times 10}\right)\\
			&= \Phi(6.81)-\Phi(3.52)\approx 0.0002325
		\end{split}
	\end{equation*}
	\section{数理统计基本概念作业}
	\begin{tcolorbox}
		[
			colframe=blue!25,
			colback=blue!10,
			coltitle=blue!20!black,  
			fonttitle=\bfseries,
			adjusted title=Formula Or Theorem:
		]
	\begin{itemize}
		\item \textbf{定义5.3.1(P232)} 设$ x_1, x_2, \cdots, x_n $为取自某总体的样本,若样本函数$ T = T(x_1, x_2, \cdots, x_n) $中不含有任何未知参数,则称$ T $为\textbf{统计量},统计量的分布称为\textbf{抽样分布}
		\item \textbf{定义5.3.2(P233)} 设$ x_1, x_2, \cdots, x_n $为取自某总体的样本,其算数平均值称为样本均值,一般用$ \bar{x} $表示,即
		\begin{equation*}
			\bar{x} = \frac{x_1 + x_2 + \cdots + x_n}{n} = \frac{1}{n}\sum\limits_{i=1}^{n}x_i
		\end{equation*}
		\item \textbf{定义5.3.3(P236)} 设$ x_1, x_2, \cdots, x_n $为取自某总体的样本,则它关于样本均值$ \bar{x} $的平均偏差平方和
		\begin{equation}
			s^2 = \frac{1}{n-1}\sum\limits_{i=1}^{n}(x_i - \bar{x})^2
		\end{equation}
		称为\textbf{样本方差}\textbf{(无偏方差)},其算数平方根$ s = \sqrt{s^2} $称为\textbf{样本标准差}
		\item \textbf{定理5.3.2(P237)} 设总体$ X $具有二阶矩,即$ E(X) = \mu $,$ Var(X) = \sigma^2 < \infty $,$ x_1, x_2, \cdots, x_n $为从该总体得到的样本,$ \bar{x} $和$ s^2 $分别是样本均值和样本方差,则
		\begin{equation*}
			\begin{split}
				E(\bar{x}) = \mu \quad &, \quad Var(\bar{x}) = \sigma^2 / n \\
				E(s^2) &= \sigma^2
			\end{split}
		\end{equation*}
	\end{itemize}
	\end{tcolorbox}
	\noindent
	3.1 \quad \textbf{A}\\
	由\textbf{定理5.3.2}可知$E(\bar{x})=\mu$、$E(s^2)=\sigma^2$
	\begin{equation*}
		\begin{split}
			&E(X) = \frac12\int_{-\infty}^{+\infty}xe^{-|x|}dx = \frac12\int_{-\infty}^{0}xe^{x}dx + \frac12\int_{0}^{+\infty}xe^{-x}dx = 0 \\
			&E(X^2) = \frac12\int_{-\infty}^{+\infty} x^2e^{-|x|}dx = \frac12\int_{-\infty}^{0} x^2e^{x}dx + \frac12\int_{0}^{+\infty}x^2e^{-x}dx = 2\\
			&Var(X) = E(X^2) -E(X)^2 = 2S
		\end{split}
	\end{equation*}
	即得$E(\bar{x})=0$、$E(s^2)=2$\\
	\\
	\\
	3.2 \quad \textbf{D} \\
	将$Y$展开
	\begin{equation*}
		\begin{split}
			Y &= \sum\limits_{i=1}^{n}\left(x_i + x_{i+n} - 2\bar{x}\right)^2 \\
			&= \sum\limits_{i=1}^{n}x_i^2 + \sum\limits_{i=1}^{n}x_{i+n}^2 + 4\sum\limits_{i=1}^{n}\bar{x}^2 + 2\sum\limits_{i=1}^{n}x_ix_{i+n} - 4\sum\limits_{i=1}^{n}x_i\bar{x} - 4\sum\limits_{i=1}^{n}x_{i+n}\bar{x} \\
			&= \sum\limits_{i=1}^{2n} x_i^2 + 4n\bar{x}^2 + 2\sum\limits_{i=1}^{n}x_ix_{i+n} -4\bar{x}\sum\limits_{i=1}^{2n}x_i \\
			&=\sum\limits_{i=1}^{2n}x_i^2 + 2\sum\limits_{i=1}^{n}x_ix_{i+n} - 4n\bar{x}^2 			
		\end{split}
	\end{equation*}
	进而可以得到
	\begin{equation*}
		\begin{split}
			E(Y) &= E\left(\sum\limits_{i=1}^{2n}x_i^2\right) + 2E\left(\sum\limits_{i=1}^{n}x_ix_{i+n}\right) - E(4n\bar{x}^2) \\
			&= 2n(\mu^2 + \sigma^2) + 2n\mu^2 - 4n(\mu^2 + \frac{\sigma^2}{2n}) \\
			&= 2(n-1)\sigma^2
		\end{split}
	\end{equation*}
	3.3 \quad \textbf{C} \\
	由$X \sim N(0, 1)$可得$\mu=0$、$\sigma^2=1$,又
	\begin{equation*}
		\begin{split}
			Cov(Y_1, Y_n) &= E(Y_1Y_n)-E(Y_1)E(Y_n) \\
			&= E\left((x_1-\bar{x})(x_n-\bar{x})\right) - E(\bar{x})^2 \\
			&= E(x_1)E(x_n) - E((x_1+x_n)\bar{x}) + E(\bar{x}^2) - E(\bar{x})^2 \\
			& = Var(\bar{x}) - E(x_1\bar{x}) - E(x_n\bar{x})
		\end{split}
	\end{equation*}
	考虑$E(x_t\bar{x})$
	\begin{equation*}
		E(x_t\bar{x}) = \frac{1}{n}E\left(\sum\limits_{i=1}^{n}x_tx_i\right) = \frac{1}{n}\sum\limits_{i=1}^{n}E(x_t)E(x_i) = \frac{n\mu^2}{n} = \mu^2\tag{*}
	\end{equation*}
	故而$Cov(Y_1, Y_n) = Var(\bar{x}) = \sigma^2/n$ \\
	\textbf{注:}$(*)$式想要说明,简单随机抽样下,样本均值与样本是独立的 \\
	\\
	3.4 \quad \textbf{D}\\
	根据\textbf{定义5.3.1}\\
	\begin{tcolorbox}
		[
		colframe=blue!25,
		colback=blue!10,
		coltitle=blue!20!black,  
		fonttitle=\bfseries,
		adjusted title=Formula Or Theorem:
		]
		\begin{itemize}
			\item \textbf{定理5.3.3(P241)}设总体$ X $的密度函数为$ p(x) $,分布函数为$ F(x) $,$ x_1, x_2, \cdots, x_n $为样本,则第$ k $个次序统计量$ x_{(k)} $的密度函数为
			\begin{equation*}
				p_k(x) = \frac{n!}{(k-1)!(n-k)!}(F(x))^{k-1}(1-F(x))^{n-k}p(x)
			\end{equation*}
			特别地
			\begin{equation*}
				\begin{split}
					p_1(x) &= n\cdot (1-F(x))^{n-1}p(x) \\
					p_n(x) &= n\cdot (F(x))^{n-1}p(x)
				\end{split}
			\end{equation*}
		\end{itemize}
	\end{tcolorbox}
	\noindent
	3.5 \quad \textbf{D} \\
	由$ X \sim U(0, 1) $,写出分布函数以及密度函数
	\begin{equation*}
		F(x) = 
		\begin{cases}
			0, & x < 0 \\
			x, & 0 \leq x < 1 \\
			1, & x \geq 1
		\end{cases} , \quad 
		p(x) = 
		\begin{cases}
			1, & 0 \leq x \leq 1 \\
			0, & others
		\end{cases}
	\end{equation*}
	由\textbf{定理5.3.3}
	\begin{equation*}
		\begin{split}
			p_1(x) = n(1-x)^{n-1} & \qquad (0 \leq x \leq 1) \\
			p_n(x) = nx^{n-1} & \qquad (0 \leq x \leq 1)
		\end{split}
	\end{equation*}
	分别求出$ E(x_{(1)}) $、$ E(x_{(n)}) $
	\begin{equation*}
		\begin{split}
			E(x_{(1)}) &= \int_{0}^{1} nx(1-x)^{n-1}dx = n\frac{\Gamma(2)\Gamma(n)}{\Gamma(n+2)} = \frac{1}{n+1} \\
			\\
			E(x_{(n)}) &= \int_{0}^{1} nx^ndx = n\frac{\Gamma(n+1)\Gamma(1)}{\Gamma(n+2)} = \frac{n}{n+1}
		\end{split}
	\end{equation*}
	因此
	\begin{equation*}
		E(x_{(n)}) - E(x_{(1)}) = \frac{n-1}{n+1}
	\end{equation*}
	\textbf{注:}P242\quad 例5.3.8 \\
	\section{三大抽样分布作业}
	\begin{tcolorbox}
		[
		colframe=blue!25,
		colback=blue!10,
		coltitle=blue!20!black,  
		fonttitle=\bfseries,
		adjusted title=Formula Or Theorem:
		]
		\begin{itemize}
			\item \textbf{定义5.4.1、5.4.2、5.4.3} \quad $ \chi^2 $分布(P250)、$ F $分布(P252)、$ t $分布(P254)
			\item \textbf{定理5.4.1(P251)}设$ x_1, x_2, \cdots, x_n $是来自正态总体$ N(\mu, \sigma^2) $的样本,其样本均值和样本方差分别为
			\begin{equation*}
				\bar{x} = \frac{1}{n}\sum\limits_{i=1}^{n} \quad , \quad s^2 = \frac{1}{n-1}\sum\limits_{i=1}^{n}(x_i-\bar{x})^2
			\end{equation*}
			则有 \\
			(1) $ \bar{x} \text{和} s^2 \text{相互独立} $ \\
			(2) $ \bar{x} \sim N(\mu, \sigma^2/n) $ \\
			(3) $ \frac{(n-1)s^2}{\sigma^2} \sim \chi^2(n-1) $
		\end{itemize}
	\end{tcolorbox}
	\noindent
	4.1 \quad\textbf{C}\\
	这里未指明$X$、$Y$相互独立\\
	\\
	4.2 \quad\textbf{A} \\
	易知
	\begin{equation*}
		\begin{split}
			X_1 - 2X_2 & \sim N(0, 20) \\
			3X_3 - 4X_4 & \sim N(0, 100) \\
		\end{split}
	\end{equation*}
	所以有
	\begin{equation*}
		\begin{split}
			\frac{X_1 - 2X_2}{\sqrt{20}} \sim N(0, 1) \qquad  &\Rightarrow\qquad \frac{(X_1 - 2X_2)^2}{20} \sim \chi^2(1) \\
			\frac{3X_3-4X_4}{\sqrt{100}} \sim N(0, 1) \qquad &\Rightarrow \qquad \frac{(3X_3-4X_4)^2}{100} \sim \chi^2(1)
		\end{split}
	\end{equation*}
	易得$ a = 1/20, b = 1/100, n = 2$\\
	\\
	4.3 \quad \textbf{B}\\
	由\textbf{定理5.3.2}可知$E(s^2)=\sigma^2=Var(X)$\\
	\begin{equation*}
		\begin{split}
			E(X) &= \int_{-\infty}^{+\infty}xf(x)dx = 0 \\
			&\\
			E(X^2) &= \frac{1}{\sqrt{2\pi}}\int_{-\infty}^{+\infty} x^4e^{-\frac{x^2}{2}}dx=\sqrt{\frac{2}{\pi}}\int_{0}^{+\infty} x^4e^{-\frac{x^2}{2}}dx \\
			&= \frac{4}{\sqrt{\pi}} \int_{0}^{+\infty} u^{\frac{3}{2}}e^{-u}du = \frac{4}{\sqrt{\pi}}\Gamma\left(\frac{5}{2}\right) = 3 \\
			& \\
			E(s^2) &= Var(X) = E(X^2) - E(X)^2 = 3
		\end{split}
	\end{equation*}
	\\
	4.4 \quad \textbf{D} \\
	\begin{equation*}
		E\left(\bar{x}^2 - \frac1n s^2\right) = E(\bar{x}^2) - \frac{1}{n}E(s^2)
	\end{equation*}
	由\textbf{定理5.4.1},可知$\bar{x}\sim N(0, 1/n)$,易得
	\begin{equation*}
		E(\bar{x}^2) = Var(\bar{x}) + E(\bar{x})^2 = \frac{1}{n}
	\end{equation*}
	而$E(s^2) = \sigma^2 = 1$,所以
	\begin{equation*}
		E\left(\bar{x}^2 - \frac1n s^2\right) = 0
	\end{equation*}
	由\textbf{定理5.4.1},可知$\bar{x}$与$s^2$相互独立,因此
	\begin{equation*}
		\begin{split}
			Var\left(\bar{x}^2 - \frac{1}{n}s^2 \right) &= Var(\bar{x}^2) + \frac{1}{n^2}Var(s^2) \\
			&= \frac{1}{n^2}Var(n\bar{x}^2) + \frac{1}{n^2(n-1)^2}Var((n-1)s^2) \\
			&= \frac{1}{n^2}Var\left(\left(\frac{\bar{x}}{1/\sqrt{n}}\right)^2\right) + \frac{1}{n^2(n-1)^2} Var\left(\frac{(n-1)s^2}{1}\right)
		\end{split}\tag{$*$}
	\end{equation*}
	由于
	\begin{equation*}
		\begin{split}
			\frac{\bar{x}}{1/\sqrt{n}} \sim N(0,1)\qquad \Rightarrow \qquad \left(\frac{\bar{x}}{1/\sqrt{n}}\right)^2 \sim \chi^2(1)
		\end{split}
	\end{equation*}
	由\textbf{定理5.4.1}可以知道
	\begin{equation*}
		\frac{(n-1)s^2}{1} \sim \chi^2(n-1)
	\end{equation*}
	因而有
	\begin{equation*}
		Var\left(\left(\frac{\bar{x}}{1/\sqrt{n}}\right)^2\right) = 2 \quad, \quad Var\left(\frac{(n-1)s^2}{1}\right) = 2(n-1)
	\end{equation*}
	带入$ (*) $即得
	\begin{equation*}
		Var\left(\bar{x}^2 - \frac{1}{n}s^2 \right) = \frac{2}{n(n-1)}
	\end{equation*}
	\\
	4.5 \quad \textbf{C} \\
	\textbf{(法1)}因为$X \sim F(n, n)$,所以$1/X \sim F(n, n)$所以有
	\begin{equation*}
		P(X<1) = P\left(\frac{1}{X} < 1\right) = P(X>1)
	\end{equation*}
	当然又有
	\begin{equation*}
		P(X<1) + P(X>1) = 1
	\end{equation*}
	因此
	\begin{equation*}
		P(X<1) = \frac12
	\end{equation*}
	\textbf{(法2)}因为$X \sim F(n, n)$,记$X$的密度函数为$p(x)$,则
	\begin{equation*}
		p(x) = \frac{\Gamma(n)}{\Gamma\left(\frac{n}{2}\right)\Gamma\left(\frac{n}{2}\right)} x^{\frac{n}{2}-1}(1+x)^{-n}
	\end{equation*}
	所以
	\begin{equation*}
		P(X<1) = \frac{\Gamma(n)}{\Gamma\left(\frac{n}{2}\right)\Gamma\left(\frac{n}{2}\right)} \int_{0}^{1} x^{\frac{n}{2}-1}(1+x)^{-n} dx\tag{$*$}
	\end{equation*}
	考虑$(*)$式的积分部分
	\begin{equation*}
		\int_{0}^{1} x^{\frac{n}{2}-1}(1+x)^{-n} dx = \int_{0}^{1} (\sqrt{x})^{n-2}(1+x)^{-n} dx \tag{$I$}
	\end{equation*}
	令$t=\sqrt{x}$,则$dx=2tdt$,带入$(I)$式
	\begin{equation*}
		(I) = \int_{0}^{1} \frac{t^{n-1} }{(1+t^2)^n} dt\tag{$ II $}
	\end{equation*}
	这里令$t=tanu$,则$dt=sec^2udu$,带入$(II)$式
	\begin{equation*}
		\begin{split}
			(II) &= 2\int_{0}^{\frac{\pi}{4}}\frac{tan^{n-1}u}{sec^{2n}u}\cdot sec^2udu = 2\int_{0}^{\frac{\pi}{4}}sin^{n-1}u \cdot cos^{n-1}u du \\
			&=\frac{1}{2^{n-2}}\int_{0}^{\frac{\pi}{4}}(2 \cdot sinu \cdot cosu)^{n-1}du = \frac{1}{2^{n-2}}\int_{0}^{\frac{\pi}{4}}sin^{n-1}2udu \\
			&= \frac{1}{2^{n-1}}\int_{0}^{\frac{\pi}{2}} sin^{n-1}vdv
		\end{split}
	\end{equation*}
	\textbf{当$2 \nmid n$时}
	\begin{equation*}
		\int_{0}^{\frac{\pi}{2}} sin^{n-1}vdv = \frac{(n-2)!!}{(n-1)!!}\frac{\pi}{2}
	\end{equation*}
	带入$(*)$式得
	\begin{equation*}
		(*) = \frac{\Gamma(n)}{\Gamma\left(\frac{n}{2}\right)\Gamma\left(\frac{n}{2}\right)} \cdot \frac{1}{2^{n-1}} \cdot \frac{(n-2)!!}{(n-1)!!}\frac{\pi}{2} = \frac{2^{n-1}\cdot (n-1)!}{((n-2)!!)^2\cdot \pi} \cdot \frac{1}{2^{n-1}} \cdot \frac{(n-2)!!}{(n-1)!!}\frac{\pi}{2} = \frac{1}{2}
	\end{equation*}
	\textbf{当$2\mid n$时},同理可得$(*) = 1/2$
	\[\qquad \]
	\textbf{注} \\
	\begin{equation*}
		I_m = \int_{0}^{\frac{\pi}{2}} cos^{m}xdx = \int_{0}^{\frac{\pi}{2}} sin^{m}xdx
	\end{equation*}
	则有
	\begin{equation*}
		I_{m} = 
		\begin{cases}
			\frac{(m-1)!!}{m!!} \cdot \frac{\pi}{2}, & 2 \mid m \\
			\frac{(m-1)!!}{m!!}, & 2 \nmid m
		\end{cases}
	\end{equation*}
	\newpage 
	\section{三大抽样分布第二次作业}
	\begin{tcolorbox}
		[
		colframe=blue!25,
		colback=blue!10,
		coltitle=blue!20!black,  
		fonttitle=\bfseries,
		adjusted title=Formula Or Theorem:
		]
		\begin{itemize}
			\item \textbf{推论5.4.1(P254)} 设$ x_1, x_2, \cdots, x_m $是来自$ N(\mu_1, \sigma_1^2) $的样本,$ y_1, y_2, \cdots, y_n $是来自$ N(\mu_2, \sigma_2^2) $的样本,且两样本相互独立,记
			\begin{equation*}
				s_x^2 = \frac{1}{m-1}\sum\limits_{i=1}^{m}(x_i - \bar{x})^2 \quad , \quad s_y^2 = \frac{1}{n-1}\sum\limits_{i=1}^{n}(y_i - \bar{y})^2
			\end{equation*}
			则有
			\begin{equation*}
				F = \frac{s_x^2 / \sigma_1^2}{s_y^2 / \sigma_2^2} \sim F(m-1, n-1)
			\end{equation*}
			特别,若$ \sigma_1^2 = \sigma_2^2 $,则$ F = s_x^2 / s_y^2 \sim F(m-1, n-1) $
			\item \textbf{推论5.4.2(P256)} 设$ x_1, x_2, \cdots, x_n $是来自正态分布$ N(\mu, \sigma^2) $的一个样本,$ \bar{x} $与$ s^2 $分别是该样本的样本均值与样本方差,则有
			\begin{equation*}
				t = \frac{\sqrt{n}(\bar{x} - \mu)}{s} \sim t(n-1)
			\end{equation*}
		\end{itemize}
	\end{tcolorbox}
	\noindent
	5.1 \quad \textbf{A} \\
	容易知道$ X $、$ Y $相互独立(P133\quad $ \rho = 0 $) \\
	\begin{equation*}
		f(x, y) = \frac{1}{12\pi} e^{-\frac{x^2}{8}}\cdot e^{-\frac{(y-1)^2}{18}} = \frac{1}{\sqrt{2\pi} \cdot 2} e^{-\frac{x^2}{2\cdot 2^2}}\cdot \frac{1}{\sqrt{2\pi} \cdot 3} e^{-\frac{(y-1)^2}{2\cdot 3^2}}
	\end{equation*}
	即可得到
	\begin{equation*}
		X \sim N(0, 4) \qquad and \qquad Y \sim N(1, 9)
	\end{equation*}
	则
	\begin{equation*}
		\left(\frac{X}{2}\right)^2 \sim \chi^2(1) \quad and \quad \left(\frac{Y-1}{3}\right)^2 \sim \chi^2(1)
	\end{equation*}
	即
	\begin{equation*}
		\frac{9}{4}\cdot \frac{X^2}{(Y-1)^2} \sim F(1, 1)
	\end{equation*}
	\\
	5.2 \quad \textbf{D} \\
	这里仅证明D选项
	\begin{equation*}
		\begin{split}
			\frac{1}{2}\sum\limits_{i=1}^{2n}X_i^2 + \sum\limits_{i=1}^{n} X_{2i-1}X_{2i} &= \frac{1}{2}\left(\sum\limits_{i=1}^{2n}X_i^2 + 2\sum\limits_{i=1}^{n} X_{2i-1}X_{2i}\right) \\
			&= \frac{1}{2}\sum\limits_{i=1}^{n}\left(X_{2i-1}^2 + 2X_{2i-1}X_{2i} + X_{2i}^2 \right) \\
			&= \sum\limits_{i=1}^{n} \left(\frac{X_{2i-1} + X_{2i}}{\sqrt{2}}\right)^2
		\end{split}
		\tag{$*$}
	\end{equation*}
	由于$ X_{2i-1} + X_{2i} \sim N(0, 2) $,所以$ (*) \sim \chi^2(n) $\\
	\\
	5.3 \quad \textbf{D} \\
	因为$X\sim N(0, \sigma^2)$,所以$\bar{x}^2$、$ s^2 $相互独立
	\begin{equation*}
		\begin{split}
			Var(\hat{\sigma}^2) &= Var(cn\bar{x}^2) + Var((1-c)s^2) \\
			&= c^2Var(n\bar{x}^2) + (1-c)^2Var(s^2) \\
			&= \sigma^4c^2Var\left(\frac{n\bar{x}^2}{\sigma^2}\right) + \frac{(1-c)^2\sigma^4}{(n-1)^2}Var\left(\frac{(n-1)s^2}{\sigma^2}\right)
		\end{split}
		\tag{$*$}
	\end{equation*}
	由\textbf{定理5.4.1}
	\begin{equation*}
		\bar{x} \sim N\left(0, \frac{\sigma^2}{n}\right) \quad and \quad \frac{(n-1)s^2}{\sigma^2} \sim \chi^2(n-1)
	\end{equation*}
	即
	\begin{equation*}
		\left(\frac{\sqrt{n}\bar{x}}{\sigma}\right)^2 \sim \chi^2(1)
	\end{equation*}
	带入$(*)$式即得
	\begin{equation*}
		(*) = 2\sigma^4c^2 + \frac{2(1-c)^2\sigma^4}{(n-1)} = \frac{2\sigma^4}{n-1}(nc^2 - 2c + 1)
	\end{equation*}
	易知,当$ c = 1/n $时,$ (*) $式取最小值\\
	\\
	5.4 \quad \textbf{A} \\
	易得
	\begin{equation*}
		\frac{\sum\limits_{k=1}^{i} X_k^2 / (\sigma^2\cdot i) }{\sum\limits_{k=i+1}^{10}X_k^2/ (\sigma^2\cdot (10-i))} = \frac{\sum\limits_{k=1}^{i} X_k^2}{\sum\limits_{k=i+1}^{10}X_k^2}\cdot \frac{10-i}{i}
	\end{equation*}
	所以
	\begin{equation*}
		\frac{10-i}{i} = 4 \quad \Rightarrow \quad i = 2
	\end{equation*}
	\\
	5.5 \quad \textbf{C} \\
	\begin{equation*}
		\begin{split}
			& P(|X|<x) = 1-P(|X|\geq x) \\
			\Rightarrow & P(|X|\geq x) = 1-\alpha \\
			=& P(X \leq -x) + P(X \geq x) = 2P(X \geq x) 
			\\
			\\
			\Rightarrow & P(X \geq x) = \frac{1-\alpha}{2}
		\end{split}
	\end{equation*}
	即 $ x = U_{(1-\alpha) / 2} $ \\
	\newpage
	\section{矩估计作业}
	\begin{tcolorbox}
		[
		colframe=blue!25,
		colback=blue!10,
		coltitle=blue!20!black,  
		fonttitle=\bfseries,
		adjusted title=Formula Or Theorem:
		]
		\begin{itemize}
			\item \textbf{替换原理与矩法估计(P272)} \\
			(1) 用样本矩去替换总体矩,这里的矩可以是原点矩也可以是中心矩 \\
			(2) 用样本矩的函数去替换相应的总体矩的函数
		\end{itemize}
	\end{tcolorbox}
	\noindent
	对于一个待估计参数$ \theta $,需要用样本均值替换总体均值;对于两个待估计参数$ \theta $、$ \lambda $,需要用样本均值替换总体均值,样本方差替换总体方差\\ \\
	6.1 \quad \textbf{D} \\
	首先求出总体均值$ E(X) $
	\begin{equation*}
		E(X) = -1 \cdot 2\theta + 0 \cdot \theta + 1 \cdot (1-3\theta) = 1 - 5 \theta
	\end{equation*}
	用样本均值$ \bar{x} $替换总体均值
	\begin{equation*}
		1 - 5\hat{\theta} = \bar{x} \quad \Rightarrow \quad \hat{\theta} = \frac{1-\bar{x}}{5}
	\end{equation*}
	\\
	6.2 \quad \textbf{D} \\
	请读者验证 \\ \\
	6.3 \quad \textbf{C} \\
	求总体均值$ E(X) $ 
	\begin{equation*}
		\begin{split}
			E(X) &= \int_{0}^{\theta} \frac{6x^2}{\theta^3}(\theta - x)dx = 6\theta \int_{0}^{\theta} \left(\frac{x}{\theta}\right)^2 \cdot \left(1-\frac{x}{\theta}\right)d\left(\frac{x}{\theta}\right) \\
			&= 6\theta \int_{0}^{1}t^2(1-t)dt = 6\theta \cdot \frac{\Gamma(3)\Gamma(2)}{\Gamma(5)} = \frac{\theta}{2} \qquad \textbf{[附录I]}
		\end{split}
	\end{equation*}
	用样本均值$ \bar{x} $替换总体均值
	\begin{equation*}
		\frac{\hat{\theta}}{2} = \bar{x} \quad \Rightarrow \quad \hat{\theta} = 2\bar{x}
	\end{equation*}
	那么
	\begin{equation*}
		Var(\hat{\theta}) = 4Var(\bar{x}) = \frac{4\sigma^2}{n}
	\end{equation*}
	需要求出$ \sigma^2 $,即$ Var(X) $
	\begin{equation*}
		\begin{split}
			E(X^2) &= \int_{0}^{\theta}\frac{6x^3}{\theta^3}(\theta - x)dx = 6\theta^2\int_{0}^{\theta} \left(\frac{x}{\theta}\right)^3\cdot \left(1-\frac{x}{\theta}\right)d\left(\frac{x}{\theta}\right) \\
			&= 6\theta^2\int_{0}^{1}t^3(1-t)^3dt = 6\theta^2\cdot \frac{\Gamma(4)\Gamma(2)}{\Gamma(6)} = \frac{3\theta^2}{10}
		\end{split}
	\end{equation*}
	\begin{equation*}
		Var(X) = E(X^2) - E(X)^2 = \frac{\theta^2}{20}
	\end{equation*}
	因此$ Var(\hat{\theta}) = \theta^2/5n $\\
	\\
	6.4 \quad \textbf{C} \\
	\textbf{利用积分求}$ E(X) $、$ Var(X) $ \\
	\begin{equation*}
		\begin{split}
			E(X) &= \frac{1}{\lambda}e^{\frac{\theta}{\lambda}}\int_{\theta}^{+\infty} xe^{-\frac{x}{\lambda}}dx = \frac{1}{\lambda}e^{\frac{\theta}{\lambda}} \cdot (-\lambda) \int_{\theta}^{+\infty} x de^{-\frac{x}{\lambda}} \\
			&= -e^{\frac{\theta}{\lambda}} \left(xe^{-\frac{x}{\lambda}} \bigg|_{\theta}^{+\infty} - \int_{\theta}^{+\infty} e^{-\frac{x}{\lambda}}dx \right) = -e^{\frac{\theta}{\lambda}} \left(-\theta e^{-\frac{\theta}{\lambda}} + \lambda e^{-\frac{x}{\lambda}} \bigg|_{\theta}^{+\infty} \right) \\
			&= \lambda + \theta
		\end{split}
	\end{equation*}
	\\
	\begin{equation*}
		\begin{split}
			E(X^2) &= \frac{1}{\lambda} e^{\frac{\theta}{\lambda}} \int_{\theta}^{+\infty} x^2e^{-\frac{x}{\lambda}}dx = \frac{1}{\lambda} e^{\frac{\theta}{\lambda}} \cdot (-\lambda)
			\int_{\theta}^{+\infty} x^2de^{-\frac{x}{\lambda}} \\
			&= -e^{\frac{\theta}{\lambda}} \left(x^2e^{-\frac{x}{\lambda}} \bigg|_{\theta}^{+\infty} - 2\int_{\theta}^{+\infty} x e^{-\frac{x}{\lambda}}dx\right) = -e^{\frac{\theta}{\lambda}} \left(-\theta^2e^{-\frac{x}{\lambda}} - 2I \right)
		\end{split}
	\end{equation*}
	由上可知
	\begin{equation*}
		\frac{1}{\lambda} e^{\frac{\theta}{\lambda}} \cdot I = \frac{1}{\lambda} e^{\frac{\theta}{\lambda}} \int_{\theta}^{+\infty} x^2e^{-\frac{x}{\lambda}}dx = \lambda + \theta 
	\end{equation*}
	即得到
	\begin{equation*}
		I = \lambda (\lambda + \theta) e^{\frac{\theta}{\lambda}}
	\end{equation*}
	带入得
	\begin{equation*}
		E(X^2) = -e^{\frac{\theta}{\lambda}} \left(-\theta^2e^{-\frac{x}{\lambda}} - 2 \lambda (\lambda + \theta) e^{-\frac{\theta}{\lambda}} \right) = \theta^2 + 2\lambda^2 + 2\lambda\theta
	\end{equation*}
	即得
	\begin{equation*}
		Var(X) = E(X^2) - E(X)^2 = \lambda^2
	\end{equation*}
	令$ \bar{x} = \lambda + \theta \quad and \quad s^2 = \lambda^2 $即可\\
	\\
	\textbf{这里复习一下4.2 特征函数求均值和方差}\\
	\textbf{求$ p(x) $特征函数}
	\begin{equation*}
		\begin{split}
			\varphi(t) &= \int_{\theta}^{+\infty} e^{itx} \cdot \frac{1}{\lambda} e^{-\frac{1}{\lambda}(x - \theta)}dx = \frac{1}{\lambda}e^{\frac{\theta}{\lambda}}\int_{\theta}^{+\infty} e^{(-\frac{1}{\lambda}+it)x}dx \\
			&= \frac{1}{\lambda} e^{-\frac{1}{\lambda}} \cdot \frac{1}{it-1/\lambda} \cdot e^{(-\frac{1}{\lambda}+it)x}\bigg|_{\theta}^{+\infty} = \frac{e^{i\theta t}}{1-i\lambda t} \\
			&\\
			\varphi'(t) &= \frac{i\theta e^{i\theta t}(1-i\lambda t) + i\lambda e^{i\theta t}}{(1-i\lambda t)^2} = e^{i\theta t} \cdot \frac{i(\lambda + \theta) + \lambda \theta t}{(1-i\lambda t)^2} \\
			&\\
			\varphi''(t) &= i\theta e^{i\theta t} \cdot \frac{i(\lambda + \theta) + \lambda \theta t}{(1-i\lambda t)^2} + e^{i\theta t} \cdot \frac{-2\lambda^2 - \lambda \theta + \lambda^2\theta ti}{(1-i\lambda t)^3}
		\end{split}
	\end{equation*}
	即有
	\begin{equation*}
		\begin{split}
			\varphi'(0) = i(\lambda + \theta) \quad and \quad 
			\varphi''(0) &= i\theta \cdot i(\lambda + \theta) + (-2\lambda^2 - \lambda \theta) \\
			& = -(\theta^2 + 2\lambda^2 + 2\lambda\theta)
		\end{split}
	\end{equation*}
	由性质\textbf{4.2.5}得
	\begin{equation*}
		\begin{split}
			E(X) &= \frac{\varphi'(0)}{i} = \lambda + \theta \\
			&\\
			Var(X) &= -\varphi''(0) + (\varphi'(0))^2 = \lambda^2
		\end{split}
	\end{equation*}
	令$ \bar{x} = \lambda + \theta \quad and \quad s^2 = \lambda^2 $即可 \\
	\\
	6.5 \quad \textbf{B} \\
	请读者验证 \\ \\
	\section{极大似然估计作业}
	\begin{tcolorbox}
		[
		colframe=blue!25,
		colback=blue!10,
		coltitle=blue!20!black,  
		fonttitle=\bfseries,
		adjusted title=Formula Or Theorem:
		]
		\begin{itemize}
			\item \textbf{定义6.3.1(P278)} 设总体的概率函数为$ p(x|\theta) $,$ \theta \in \Theta $,其中$ \theta $是一个未知参数或几个未知参数组成的参数向量,$ \Theta $是参数空间,$ x_1, x_2, \cdots, x_n $是来自该总体的样本,将样本的联合概率函数看成$ \theta $的函数,用$ L(\theta|x_1, x_2, \cdots, x_n) $表示,简记为$ L(\theta) $,
			\begin{equation*}
				L(\theta) = L(\theta|x_1, x_2, \cdots, x_n) = p(x_1|\theta)P(x_2|\theta)\cdots p(x_n|\theta)
			\end{equation*}
			$ L(\theta) $称为\textbf{似然函数}.如果某统计量$ \hat{\theta} = \hat{\theta}(x_1, x_2, \cdots, x_n) $满足
			\begin{equation*}
				L(\hat{\theta}) = \max\limits_{\theta \in \Theta} L(\theta)
			\end{equation*}
			则称$ \hat{\theta} $是$ \theta $的\textbf{最大似然估计}
			\item In brief, a statistic $ \hat{\theta}(x_1, x_2, \cdots, x_n) $ is a \textbf{maximum likelihood estimator} of $ \theta $ if, for each sample $ x_1, x_1, \cdots, x_n, \hat{\theta}(x_1, x_2, \cdots, x_n) $ is a value for the parameter that maximizes the likelihood function $ L(\theta|x_1, x_2, \cdots, x_n) $
		\end{itemize}
	\end{tcolorbox}
	\newpage 
	\noindent
	7.1 \quad \textbf{C} \\
	对$ F(x|\theta) $求导得到密度函数
	\begin{equation*}
		p(x|\theta) = 
		\begin{cases}
			\frac{2\theta}{x^3}, & x \geq \theta (\theta > 1) \\
			0, & others
		\end{cases}
	\end{equation*}
	写出似然函数
	\begin{equation*}
		L(\theta) = \prod\limits_{i=1}^{n} p(x|\theta) = \prod\limits_{i=1}^{n} \frac{2\theta}{x_i^3} \qquad (x_i \geq \theta)
	\end{equation*}
	两边取对数得
	\begin{equation*}
		lnL(\theta) = nln2\theta - \sum\limits_{i=1}^{n}lnx_i^3
	\end{equation*}
	显然$ L(\theta) $单调递增,又有$ x_i \geq \theta $,所以$ \hat{\theta} = min\{x_i\} = x_{(1)} $ \\
	\\
	7.2 \quad \textbf{B} \\
	写出似然函数
	\begin{equation*}
		L(\theta) = \prod\limits_{i=1}^{n} (1+\theta)x_i^{\theta}
	\end{equation*}
	两边取对数得
	\begin{equation*}
		lnL(\theta) = nln(1+\theta) + \theta \sum\limits_{i=1}^{n} lnx_i
	\end{equation*}
	对$ \theta $求导
	\begin{equation*}
		\frac{d}{d\theta} lnL(\theta) = \frac{n}{1+\theta} + \sum\limits_{i=1}^{n}lnx_i
	\end{equation*}
	令导数等于0,求导极大值点为
	\begin{equation*}
		\hat{\theta} = -\frac{n}{\sum\limits_{i=1}^{n}lnx_i} - 1
	\end{equation*}
	\\
	7.3 \quad \textbf{B} \\
	请读者验证 \\
	\\ 
	7.4 \quad \textbf{C} \\
	求出总体均值
	\begin{equation*}
		E(X) = -\theta^2 + (1-\theta)^2 = 1 - 2\theta 
	\end{equation*}
	用样本均值替换总体均值
	\begin{equation*}
		1 - 2\theta = \bar{x} \quad \Rightarrow \quad \bar{\theta}_M = \frac{1 - \bar{x}}{2}
	\end{equation*}
	写出似然函数
	\begin{equation*}
		\begin{split}
			L(\theta) &= \theta^{2N_1} \cdot (2\theta(1-\theta))^{N_2} \cdot (1-\theta)^{2(n-N_1-N_2)} \\
			&= 2^{N_2} \cdot \theta^{2N_1+N_2} \cdot (1-\theta)^{2n-2N_1-N_2}
		\end{split}
	\end{equation*}
	令$ s = 2N_1 + N_2 $,$ t = 2n - 2N_1 - N_2 $ ,对$ \theta $求导
	\begin{equation*}
		L(\theta) = 2^{N_2} \cdot \theta^{s} \cdot (1-\theta)^{t}
	\end{equation*}
	\begin{equation*}
		\frac{d}{d\theta}L(\theta) = 2^{N_2} \cdot \theta^{s-1}(1-\theta)^{t-1}(s-(s+t)\theta)
	\end{equation*}
	求出极大值点为
	\begin{equation*}
		\hat{\theta}_L = \frac{s}{s + t} = \frac{2N_1 + N_2}{2n}
	\end{equation*}
	\\
	7.5 \quad \textbf{A} \\
	求出总体均值
	\begin{equation*}
		E(X) = \sum\limits_{t=1}^{N} \frac{t}{N} = \frac{1+N}{2}
	\end{equation*}
	用样本均值替换总体均值
	\begin{equation*}
		\frac{1+N}{2} = \bar{x} \quad \Rightarrow \quad \hat{N}_M = 2\bar{x} - 1
	\end{equation*}
	写出似然函数
	\begin{equation*}
		L(N) = \prod\limits_{i=1}^{N} \frac{1}{N} = \left(\frac{1}{N}\right)^N \qquad (x_i \leq N)
	\end{equation*}
	显然$ L(N) $单调递减,又有$ x_i \leq N $,所以$ \hat{N}_L = max\{x_i\} = x_{(n)} $ \\ \\
	\section{估计量的评判标准作业} 
	\begin{tcolorbox}
		[
		colframe=blue!25,
		colback=blue!10,
		coltitle=blue!20!black,  
		fonttitle=\bfseries,
		adjusted title=Formula Or Theorem:
		]
		\begin{itemize}
			\item \textbf{定义6.2.1(P267)} 设$ \hat{\theta} = \hat{\theta}(x_1, x_2, \cdots, x_n) $是$ \theta $的一个估计,$ \theta $的参数空间为$ \Theta $,若对任意的$ \theta \in \Theta $,有
			\begin{equation*}
				E(\hat{\theta}) = \theta
			\end{equation*}
			则称$ \hat{\theta} $是$ \theta $的\textbf{无偏估计},否则称为\textbf{有偏估计}
			\item A statistic $ \hat{\theta} $ is said to be an \textbf{unbiased estimator}, or its value an unbiased estimate, if and only if the mean of the sampling distribution of the estimator $ E(\hat{\theta}) = \theta $, whatever the value of $ \theta $
		\end{itemize}
	\end{tcolorbox}
	\noindent
	8.1 \quad \textbf{C} \\
	因为$ x_i \sim N(\mu, \sigma^2) $,容易知道
	\begin{equation*}
		x_{i+1} - x_{i} \sim N(0, 2\sigma^2)
	\end{equation*}
	标准化后得
	\begin{equation*}
		\frac{x_{i+1} - x_{i}}{\sqrt{2} \sigma} \sim N(0, 1) \quad \Rightarrow \quad \left(\frac{x_{i+1} - x_{i}}{\sqrt{2} \sigma}\right)^2 \sim \chi^2(1)
	\end{equation*}
	求和得
	\begin{equation*}
		\sum\limits_{i=1}^{n-1} \left(\frac{x_{i+1} - x_{i}}{\sqrt{2} \sigma}\right)^2 \sim \chi^2(n-1)
	\end{equation*}
	求数学期望得
	\begin{equation*}
		\frac{1}{2\sigma^2} E \left(\sum\limits_{i=1}^{n-1} \left(x_{i+1} - x_{i}\right)^2\right) = n - 1
	\end{equation*}
	即得
	\begin{equation*}
		cE \left(\sum\limits_{i=1}^{n-1} \left(x_{i+1} - x_{i}\right)^2\right) = 2c\sigma^2(n-1) 
	\end{equation*}
	因为目标统计量是$ \sigma^2 $的无偏估计,可知
	\begin{equation*}
		2c\sigma^2(n-1) = \sigma^2 \quad \Rightarrow \quad c = \frac{1}{2(n-1)}
	\end{equation*}
	\\
	8.2 \quad \textbf{A} \\
	求数学期望
	\begin{equation*}
		E(\bar{x} + cs^2) = E(\bar{x}) + cE(s^2) = np(1+c-cp)
	\end{equation*}
	由\textbf{定义6.2.1}
	\begin{equation*}
		1+c-cp = p \quad \Rightarrow \quad c = -1
	\end{equation*}
	\\
	8.3 \quad \textbf{D} \\
	这里仅证明D选项
	先求分布函数
	\begin{equation*}
		F(x) = 
		\begin{cases}
			1 - e^{-(x-\theta)}, & x \geq \theta \\
			0, & x < \theta
		\end{cases}
	\end{equation*}
	根据\textbf{定理5.3.3}
	\begin{equation*}
		p_1(x) = n \cdot (1-F(x))^{n-1} p(x)
	\end{equation*}
	带入得到
	\begin{equation*}
		p_1(x) = n e^{-n(x-\theta)}
	\end{equation*}
	求$ x_{(1)} $的数学期望
	\begin{equation*}
		\begin{split}
			E(x_{(1)}) &= \int_{\theta}^{+\infty} nx e^{-n(x -\theta)}dx = \int_{0}^{+\infty} n(t+\theta)e^{-nt}dt \\
			&= \frac{1}{n} \int_{0}^{+\infty} nt e^{-nt}d(nt) + n\theta \int_{0}^{+\infty} e^{-nt}dt \\
			&= \frac{1}{n} \Gamma(2) + \theta = \frac{1}{n} + \theta
		\end{split}
	\end{equation*}
	所以$ E(x_{(1)} - 1/n) = \theta $ \\
	\\
	8.4 \quad \textbf{C} \\
	直接求$ E(\hat{p}) $
	\begin{equation*}
		\begin{split}
			E(\hat{p}) &= c E\left(\sum\limits_{i=1}^{n} x_i(x_i - 1)\right) = c \sum\limits_{i=1}^n\left(E(x_i^2) - E(x_i)\right) \\
			&= c \sum\limits_{i=1}^n \left(Var(x_i) + E(x_i)^2 - E(x_i)\right) = c\sum\limits_{i=1}^n(mp(1-p) + m^2p^2 - mp) \\
			&= cnp^2m(m-1)
		\end{split}
	\end{equation*}
	由于$ \hat{p} $ 是$ p^2 $的无偏估计,根据\textbf{定义6.2.1}
	\begin{equation*}
		cnp^2m(m-1) = p^2 \quad \Rightarrow \quad c = \frac{1}{mn(m-1)}
	\end{equation*} \\
	\begin{tcolorbox}
		[
		colframe=blue!25,
		colback=blue!10,
		coltitle=blue!20!black,  
		fonttitle=\bfseries,
		adjusted title=Formula Or Theorem:
		]
		\begin{itemize}
			\item \textbf{定义6.1.3} 设$ \hat{\theta_1} $,$ \hat{\theta_2} $是$ \theta $的两个无偏估计,如果对任意的$ \theta \in \Theta$有
			\begin{equation*}
				Var(\hat{\theta_1}) \leq Var(\hat{\theta_2})
			\end{equation*}
			且至少有一个$ \theta \in \Theta $使得上述不等号严格成立,则称$ \hat{\theta_1} $比$ \hat{\theta_2} $有效
			\item A statistic $ \hat{\theta_1} $ is said to be a \textbf{more efficient unbiased estimator} of the parameter $ \theta $ than the statistic $ \hat{\theta_2} $ if \\
			(1) $ \hat{\theta_1} $ and $ \hat{\theta_2} $ are both unbiased estimators of $ \theta $ \\
			(2) the variance of the sampling distribution of the first estimator is no larger than that of the second and is smallar for at least one value of $ \theta $
		\end{itemize}
	\end{tcolorbox}
	\noindent
	8.5 \quad \textbf{C} \\
	因为$ X \sim U(0, \theta) $,写出总体的分布函数以及密度函数
	\begin{equation*}
		F(x) = 
		\begin{cases}
			0, & x<0 \\
			\frac{x}{\theta} , & 0 \leq x < \theta \\
			1, & x \geq \theta
		\end{cases}
		\quad , \quad 
		p(x) = 
		\begin{cases}
			\frac{1}{\theta} , & 0 \leq x \leq \theta \\
			0 , & others
		\end{cases}
	\end{equation*}
	求出$ x_{(1)} $的密度函数,根据\textbf{定理5.3.3}
	\begin{equation*}
		p_1(x) = n\cdot (1-F(x))^{n-1}p(x) = n\left(1-\frac{x}{\theta}\right)^{n-1} \frac{1}{\theta}
	\end{equation*}
	求出$ E(x_{(1)}) $
	\begin{equation*}
		\begin{split}
			E(x_{(1)}) &= \frac{n}{\theta} \int_{0}^{\theta} x \left(1 - \frac{x}{\theta}\right)^{n-1} dx = \theta n \int_{0}^{\theta} \frac{x}{\theta} \cdot \left(1 - \frac{x}{\theta}\right)^{n-1} d \left(\frac{x}{\theta}\right) \\
			&= \theta n \int_{0}^{1} t(1-t)^{n-1}dt = \theta n \frac{\Gamma(2)\Gamma(n)}{\Gamma(n+2)} = \frac{\theta}{n+1} \qquad \textbf{[附录I]}
		\end{split}
	\end{equation*}
	求出$ Var(x_{(1)}) $
	\begin{equation*}
		\begin{split}
			E(x_{(1)}^2) &= \frac{n}{\theta} \int_{0}^{\theta} x^2 \left(1 - \frac{x}{\theta}\right)^{n-1} dx = \theta^2 n \int_{0}^{\theta} \left(\frac{x}{\theta}\right)^2 \cdot \left(1 - \frac{x}{\theta}\right)^{n-1} d \left(\frac{x}{\theta}\right) \\
			&= \theta^2 n \int_{0}^{1} t^2 (1-t)^{n-1}dt = \theta^2 n \frac{\Gamma(3)\Gamma(n)}{\Gamma(n+3)} = \frac{2\theta^2}{(n+2)(n+1)}
		\end{split}
	\end{equation*}
	\begin{equation*}
		Var(x_{(1)}) = E(x_{(1)}^2) - E(x_{(1)})^2 = \frac{n\theta^2}{(n+2)(n+1)^2}
	\end{equation*}
	求出$ x_{(n)} $的密度函数,根据\textbf{定理5.3.3}
	\begin{equation*}
		p_n(x) = n \cdot (F(x))^{n-1} p(x) = n \left(\frac{x}{\theta}\right)^{n-1} \frac{1}{\theta}
	\end{equation*}
	求出$ E(x_{(n)}) $
	\begin{equation*}
		\begin{split}
			E(x_{(n)}) &= \frac{n}{\theta} \int_{0}^{\theta} x\left(\frac{x}{\theta}\right)^{n-1}dx = \theta n \int_{0}^{\theta} \left(\frac{x}{\theta}\right)^{n} d\left(\frac{x}{\theta}\right) \\
			&= \theta n \int_{0}^{1} t^n (1-t)^0 dt = \theta n \frac{\Gamma(n+1)\Gamma(1)}{\Gamma(n+2)} = \frac{n \theta}{n + 1}
		\end{split}
	\end{equation*}
	求出$ Var(x_{(n)}) $
	\begin{equation*}
		\begin{split}
			E(x_{(n)}^2) &= \frac{n}{\theta} \int_{0}^{\theta} x^2 \left(\frac{x}{\theta}\right)^{n-1}dx = \theta^2 n \int_{0}^{\theta} \left(\frac{x}{\theta}\right)^{n+1} d\left(\frac{x}{\theta}\right) \\
			&= \theta^2 n \int_{0}^{1} t^{n+1}(1-t)^0 dt = \theta^2 n \frac{\Gamma(n+2)\Gamma(1)}{\Gamma(n+3)} = \frac{\theta^2 n}{n+2}
		\end{split}
	\end{equation*}
	\begin{equation*}
		Var(x_{(n)}) = E(x_{(n)}^2) - E(x_{n})^2 = \frac{n\theta^2}{(n+2)(n+1)^2}
	\end{equation*}
	求$ E(\hat{\theta_1}) $ 与 $ E(\hat{\theta_2}) $
	\begin{equation*}
		\begin{split}
			E(\hat{\theta_1}) &= \frac{n+1}{n} \cdot \frac{n\theta}{n+1} = \theta \\
			&\\
			E(\hat{\theta_2}) &= (n+1) \cdot \frac{\theta}{n+1} = \theta
		\end{split}
	\end{equation*}
	所以$ \hat{\theta_1} $ 与 $ \hat{\theta_2} $ 都是 $ \theta $的无偏估计,再求$ Var(\hat{\theta_1}) $和$ Var(\hat{\theta_2}) $
	\begin{equation*}
		\begin{split}
			Var(\hat{\theta_1}) &= \left(\frac{n+1}{n}\right)^2 \cdot \frac{n\theta^2}{(n+2)(n+1)^2} = \frac{\theta^2}{n+2} \\
			\\
			Var(\hat{\theta_2}) &= (n+1)^2 \cdot \frac{n\theta^2}{(n+2)(n+1)^2} = \frac{n\theta^2}{n+2}
		\end{split}
	\end{equation*}
	显然$ Var(\hat{\theta_1}) < Var(\hat{\theta_2})$,所以$ \hat{\theta_1} $更有效 \\ \\
	8.6 \quad \textbf{D} \\
	由\textbf{定理5.4.1}
	\begin{equation*}
		\bar{x} \sim N(0, \sigma^2 / n) \quad \Rightarrow \quad \frac{\bar{x}}{\sigma / \sqrt{n}} \sim N(0, 1)
	\end{equation*}
	即有
	\begin{equation*}
		\left(\frac{\bar{x}}{\sigma^2 / \sqrt{n}}\right)^2 \sim \chi^2(1)
	\end{equation*}
	所以
	\begin{equation*}
		E(\bar{x}^2) = \frac{n}{\sigma^2}
	\end{equation*}
	由\textbf{定理5.4.1}
	\begin{equation*}
		\frac{x_n}{\sigma} \sim N(0, 1) \quad \Rightarrow \quad \left(\frac{x_n}{\sigma}\right)^2 \sim \chi^2(1)
	\end{equation*}
	所以
	\begin{equation*}
		E(x_n) = \sigma^2 \quad , \quad Var(x_n) = 2 \sigma^4
	\end{equation*}
	由\textbf{定理5.4.1}
	\begin{equation*}
		\frac{(n-1)s^2}{\sigma^2} \sim \chi^2(n-1)
	\end{equation*}
	所以
	\begin{equation*}
		E(s^2) = \sigma^2 \quad , \quad Var(s^2) = \frac{2 \sigma^4}{n-1}
	\end{equation*}
	由\textbf{定理5.4.1}
	\begin{equation*}
		\frac{x_i}{\sigma} \sim N(0, 1) \quad \Rightarrow \quad \left(\frac{x_i}{\sigma}\right)^2 \sim \chi^2(1)
	\end{equation*}
	即有
	\begin{equation*}
		\sum\limits_{i=1}^{n} \left(\frac{x_i}{\sigma}\right)^2 \sim \chi^2(n)
	\end{equation*}
	所以
	\begin{equation*}
		E\left(\frac{1}{n} \sum\limits_{i=1}^{n} x_i^2 \right) = \sigma^2 \quad , \quad Var\left(\frac{1}{n} \sum\limits_{i=1}^{n} x_i^2 \right) = \frac{2 \sigma^4}{n}
	\end{equation*}
	因此$ \sum\limits_{i=1}^{n} x_i^2 / n $是最有效的无偏估计量 \\ \\
	8.7 \quad \textbf{D} \\
	容易知道
	\begin{equation*}
		T \sim N((4a+9b)\mu, 4a^2+9b^2)
	\end{equation*}
	因为$ T $是$ \mu $的无偏估计,所以
	\begin{equation*}
		4a + 9b = 1
	\end{equation*}
	当$ 4a^2+9b^2 $最小时,$ T $最有效,带入即可知道$ a = 4/25, b= 1/ 25 $ \\ \\
	8.8 \quad \textbf{B} \\
	请读者验证
	\newpage
	\section{区间估计作业}
	\begin{tcolorbox}
		[
		colframe=blue!25,
		colback=blue!10,
		coltitle=blue!20!black,  
		fonttitle=\bfseries,
		adjusted title=Formula Or Theorem:
		]
		\begin{itemize}
			\item \textbf{定义6.6.1(P300)} 设$ \theta $是总体的一个参数,其参数空间为$ \Theta $,$ x_1, x_2, \cdots, x_n $是来自该总体的样本,对给定的一个$ \alpha (0 < \alpha < 1)$,假设有两个统计量$ \hat{\theta_L} = \hat{\theta_L}(x_1, x_2, \cdots, x_n) $和$ \hat{\theta_U} = \hat{\theta_U}(x_1, x_2, \cdots, x_n) $,若对任意的$ \theta \in \Theta $,有
			\begin{equation*}
				P_{\theta} (\hat{\theta_L} \leq \theta \leq \hat{\theta_U}) \geq 1 - \alpha 
			\end{equation*}
			则称随机区间$ [\hat{\theta_L}, \hat{\theta_U}] $为$\theta$的\textbf{置信水平为$ 1-\alpha $的置信区间},$ \hat{\theta_L} $和$ \hat{\theta_U} $分别称为$\theta$的\textbf{置信下限}和\textbf{置信上限}
			\item \textbf{Large sample confidence interval for $ \mu $ ($ \sigma $ konwn)} 
			\begin{equation*}
				\bar{x} - u_{1 - \alpha / 2} \cdot \frac{\sigma}{\sqrt{n}} < \mu < \bar{x} + u_{1 - \alpha / 2} \cdot \frac{\sigma}{\sqrt{n}}
			\end{equation*}
		\end{itemize}
	\end{tcolorbox}
	\noindent
	9.1 \quad \textbf{A} \\
	由题意得
	\begin{equation*}
		\begin{cases}
			\bar{x} - u_{1 - \alpha / 2} \cdot \frac{\sigma}{\sqrt{n}} &= 43.88 \\
			\bar{x} + u_{1 - \alpha / 2} \cdot \frac{\sigma}{\sqrt{n}} &= 46.52
		\end{cases}
	\end{equation*}
	消去$ \bar{x} $
	\begin{equation*}
		1.6 u_{1 - \alpha / 2} = 2.64 \quad \Rightarrow \quad u_{1 - \alpha / 2} = 1.65
	\end{equation*}
	即得
	\begin{equation*}
		\Phi(1.65) = 0.9505 = 1 - \frac{\alpha}{2} \quad \Rightarrow \quad \alpha = 0.10
	\end{equation*}
	置信水平$ 1 - \alpha = 0.90 $ \\ \\
	9.2 \quad \textbf{C} \\
	由题意得
	\begin{equation*}
		E = u_{1-\alpha/2} \cdot \frac{\sigma}{\sqrt{n}} = u_{0.975}\cdot\frac{4}{\sqrt{n}} < 2
	\end{equation*}
	即得
	\begin{equation*}
		\sqrt{n} \geq 2u_{0.975} \quad \Rightarrow \quad n \geq 16
	\end{equation*} \\
	9.3 \quad \textbf{B} \\
	由题意得
	\begin{equation*}
		E = u_{1-\alpha/2} \cdot \frac{\sigma}{\sqrt{n}} = u_{0.975}\frac{0.3}{\sqrt{15}}
	\end{equation*}
	置信区间为
	\begin{equation*}
		I = [\bar{x}-E, \bar{x}+E] = [2.88-E, 2.88+E]
	\end{equation*}
	9.4 \quad \textbf{B} \\
	请读者验证
	\newpage
	\section{区间估计第二次作业} 
	\begin{tcolorbox}
		[
		colframe=blue!25,
		colback=blue!10,
		coltitle=blue!20!black,  
		fonttitle=\bfseries,
		adjusted title=Formula Or Theorem:
		]
		\begin{itemize}
			\item \textbf{Small sample confidence interval for $ \mu $ of normal population($\sigma$ unknown)}
			\begin{equation*}
				\bar{x} - t_{1-\alpha/2} \cdot \frac{s}{\sqrt{n}} < \mu < \bar{x} + t_{1-\alpha/2} \cdot \frac{s}{\sqrt{n}}
			\end{equation*}
			\item \textbf{Confidence interval of $ \sigma^2 $ when $ \mu $ unknown(P305)} 
			\begin{equation*}
				\frac{(n-1)s^2}{\chi_{1-\alpha/2}^{2}}< \sigma^2 < \frac{(n-1)s^2}{\chi_{\alpha/2}^{2}}
			\end{equation*}
		\end{itemize}
	\end{tcolorbox}
	\noindent
	10.1 \quad \textbf{D} \\
	$\sigma$未知,考虑使用$ \sqrt{n} (\bar{x} - \mu) / s $ 作为枢轴量,$\mu$的置信区间为
	\begin{equation*}
		[\bar{x} - t_{1-\alpha/2} \cdot \frac{s}{\sqrt{n}},\quad \bar{x} + t_{1-\alpha/2} \cdot \frac{s}{\sqrt{n}}]
	\end{equation*}
	所以$ e^{\mu} $的置信区间为
	\begin{equation*}
		\left[e^{\bar{x} - t_{1-\alpha/2} \cdot \frac{s}{\sqrt{n}}}, \quad e^{\bar{x} + t_{1-\alpha/2} \cdot \frac{s}{\sqrt{n}}}\right]
	\end{equation*}
	\\
	10.2 \quad \textbf{A} \\
	已知$ lnX \sim N(\mu, 1) $,\textbf{断言(*)}$ E(X) = e^{\mu + \frac{1}{2}} $,所以可以考虑$\mu$的置信区间($\sigma$已知) \\ \\
	记$ Y = lnX \sim N(\mu, 1) $,则
	\begin{equation*}
		P\left(\bigg|\frac{\bar{y}-\mu}{\sigma/\sqrt{n}}\bigg| < u_{1-\alpha/2} \right) = 1 - \alpha
	\end{equation*}
	即$\mu$的置信区间为
	\begin{equation*}
		\bar{y} - u_{1-\alpha/2} \cdot \frac{\sigma}{\sqrt{n}} < \mu < \bar{y} + u_{1-\alpha/2} \cdot \frac{\sigma}{\sqrt{n}}
	\end{equation*}
	其中$ \bar{y} = 0 $, $ u_{0.975} = 1.96 $,$ \sigma = 1 $,$ n = 4 $,则
	\begin{equation*}
		-0.98 < \mu < 0.98
	\end{equation*}
	所以
	\begin{equation*}
		e^{-0.48} < E(X) < e^{1.48}
	\end{equation*}
	\textbf{证明(*)处的一个结论(P115 习题2.6.13)} 设随机变量$ X \sim N(\mu, \sigma^2) $,求$ Y = e^X $的数学期望与方差 \\
	求出$ Y $的分布(Y > 0)
	\begin{equation*}
		F_Y(y) = P(Y \leq y) = P(e^X \leq y) = P(X \leq lny) = F_X(lny)
	\end{equation*}
	所以
	\begin{equation*}
		p_y(y) = \frac{d}{dy}F_Y(y) = \frac{1}{y} \cdot \frac{d}{dy}F_X(lny) = \frac{1}{\sqrt{2\pi}\sigma} = \frac{1}{\sqrt{2\pi}\sigma y} e^{-\frac{(lny-\mu)^2}{2\sigma^2}} \qquad (y > 0)
	\end{equation*}
	求$ E(Y) $
	\begin{equation*}
		E(Y) = \frac{1}{\sqrt{2\pi}\sigma} \int_{0}^{+\infty} e^{-\frac{(lny-\mu)^2}{2\sigma^2}} dy
	\end{equation*}
	令$ (lny - \mu) / \sigma = t $,得到
	\begin{equation*}
		\begin{split}
			E(Y) &= \frac{1}{\sqrt{2\pi}\sigma} \int_{-\infty}^{+\infty} e^{-\frac{t^2}{2}} \cdot e^{\sigma t + \mu}\sigma dt = \frac{e^{\mu}}{\sqrt{2\pi}} \int_{-\infty}^{+\infty} e^{-\frac{1}{2}(t^2-2\sigma t)} dt \\
			&= e^{\mu + \frac{1}{2} \sigma^2} \cdot \frac{1}{\sqrt{2\pi}} \int_{-\infty}^{+\infty} e^{-\frac{(t-\sigma)^2}{2}}dt = e^{\mu + \frac{1}{2}\sigma^2} 
		\end{split}
	\end{equation*}
	求$ Var(X) $
	\begin{equation*}
		\begin{split}
			E(Y^2) &= \frac{1}{\sqrt{2\pi}\sigma} \int_{0}^{+\infty} y e^{-\frac{(lny - \mu)^2}{2\sigma^2}}dy = e^{2\mu + 2\sigma^2} \cdot \frac{1}{\sqrt{2\pi}} \int_{-\infty}^{+\infty} e^{-\frac{(t-2\sigma)^2}{2}}dt \\
			&= e^{2\mu + 2\sigma^2} \\
			& \\
			Var(Y) &= E(Y^2) -E(Y)^2 = e^{2\mu + \sigma^2} (e^{\sigma^2} - 1)
		\end{split}
	\end{equation*}
	\textbf{题目中}$ Y = lnX \sim N(\mu, 1) $,所以$ X = e^{Y} $,则$ E(X) = e^{\mu + \frac{1}{2}} $
	\\ \\
	10.3 \quad \textbf{C} \\
	显然,读者自证 \\ \\
	10.4 \quad \textbf{B} \\
	$ \mu $未知,考虑使用$ (n-1)s^/\sigma^2 $作为枢轴量,$\sigma^2$的置信区间为
	\begin{equation*}
		\frac{(n-1)s^2}{\chi_{1-\alpha/2}^{2}}< \sigma^2 < \frac{(n-1)s^2}{\chi_{\alpha/2}^{2}}
	\end{equation*}
	带入$ n = 9 $,$ s^2 = 11 $,$ \alpha = 0.5 $即可 \\
	\\
	\section{单侧置信区间作业}
	\noindent
	11.1 \quad \textbf{A} \\
	选定枢轴量
	\begin{equation*}
		\frac{\bar{x} - \mu}{s / \sqrt{n}} \sim t(n - 1)
	\end{equation*}
	则
	\begin{equation*}
		P\left( \frac{\bar{x} - \mu}{s / \sqrt{n}} \leq  t_{1 - \alpha}(n - 1)\right) = 1 - \alpha
	\end{equation*}
	即
	\begin{equation*}
		P\left(\mu \geq \bar{x} - t_{1 - \alpha}(n - 1) \frac{s}{\sqrt{n}}\right) = 1 - \alpha
	\end{equation*}
	所以$ \mu $的置信下限为$ \bar{x} - t_{1 - \alpha}(n - 1) \cdot s / \sqrt{n} $ \\ \\
	11.2 \quad \textbf{D} \\
	选定枢轴量
	\begin{equation*}
		\frac{\bar{x} - \mu}{s / \sqrt{n}} \sim t(n - 1)
	\end{equation*}
	则
	\begin{equation*}
		P\left( \frac{\bar{x} - \mu}{s / \sqrt{n}} \geq  t_{\alpha}(n - 1)\right) = 1 - \alpha
	\end{equation*}
	即
	\begin{equation*}
		P\left(\mu \leq \bar{x} - t_{\alpha}(n - 1) \frac{s}{\sqrt{n}}\right) = 1 - \alpha
	\end{equation*}
	所以$ \mu $的置信上限为$ \bar{x} - t_{\alpha}(n - 1) \cdot s / \sqrt{n} $ \\ \\
	\textbf{注:}$ t_{\alpha} = - t_{1 - \alpha} $ \qquad 读者自证不难\\ \\
	11.3 \quad \textbf{D} \\
	易知
	\begin{equation*}
		\bar{x}_1 - \bar{x}_2 \sim N \left(\mu_1 - \mu_2, \frac{\sigma_1^2 + \sigma_2^2}{n} \right)
	\end{equation*}
	则选定枢轴量为
	\begin{equation*}
		\frac{(\bar{x}_1 - \bar{x}_2) - (\mu_1 - \mu_2)}{\sqrt{(\sigma_1^2 + \sigma_2^2) / n}} \sim N(0, 1)
	\end{equation*}
	则有
	\begin{equation*}
		P \left( \frac{(\bar{x}_1 - \bar{x}_2) - (\mu_1 - \mu_2)}{\sqrt{(\sigma_1^2 + \sigma_2^2) / n}} \geq u_{\alpha} \right) = 1 - \alpha
	\end{equation*}
	即
	\begin{equation*}
		P \left(\mu_1 - \mu_2 \leq (\bar{x}_1 - \bar{x}_2) - u_{\alpha} \cdot \sqrt{\frac{\sigma_1^2 + \sigma_2^2}{n}}\right) = 1 - \alpha
	\end{equation*}
	所以$ \mu_1 - \mu_2 $的置信上限为$ (\bar{x}_1 - \bar{x}_2) - u_{\alpha} \cdot \sqrt{(\sigma_1^2 + \sigma_2^2) / n} $ \\ \\
	\textbf{注:}$ u_{\alpha} = - u_{1 - \alpha} $ \qquad 读者自证不难\\ \\
	11.4 \quad \textbf{B} \\
	显然$ \bar{x} \pm u_{1-\alpha / 2} \cdot \sigma / \sqrt{n} $ \\ \\ 
	\newpage
	\section{假设检验的思想及单正态总体检验}
	\begin{tcolorbox}
		[
		colframe=blue!25,
		colback=blue!10,
		coltitle=blue!20!black,  
		fonttitle=\bfseries,
		adjusted title=Formula Or Theorem:
		]
		\begin{itemize}
			\item \textbf{假设检验的基本步骤(P315)} \\
			(1) \quad 建立假设 $ (H_0: \quad vs \quad H_1:) $\\
			(2) \quad 选择检验统计量(t or u) \\
			(3) \quad 选择显著性水平 $ \alpha =  $\\
			(4) \quad 给出拒绝域 $ W = $\\
			(5) \quad 做出判断 (reject or not reject)
			\item \textbf{some definitions} \\
			\textbf{$ H_0 $}: Null hypothesis \\
			\textbf{$ H_1 $}: Alternative hypothesis \\
			\textbf{Type I error}: Rejection of $ H_0 $ when $ H_0 $ is true \\
			\textbf{Type II error}: Nonrejection of $ H_0 $ when $ H_1 $ is true \\
			$ \alpha $: probability of making a Type I error(also called the level of significance) \\
			$ \beta $: probability of making a Type II error
		\end{itemize}
	\end{tcolorbox}
	\noindent
	12.1 \quad \textbf{C} \\
	definition of Type II error \\ \\
	12.2 \quad \textbf{D} \\
	1. Null hypothesis: $ \sigma \leq \sigma_0 = 0.50 $ \\
	Alternative hypothesis: $ \sigma > \sigma_0 = 0.50 $ \\
	2. Level of significance: $ \alpha = 0.05 $ \\
	3. Criterion: \\
	test statistic:
	\begin{equation*}
		\chi^2 = \frac{(n-1)s^2}{\sigma_0^2} \sim \chi^2(n-1)
	\end{equation*}
	rejection regions:
	\begin{equation*}
		W = \{\chi^2 > \chi_{1-\alpha}^2(n-1)\} \\
	\end{equation*}
	4. Calculations:
	\begin{equation*}
		\chi^2 = \frac{(25 - 1) \cdot 0.58^2}{0.50^2} = 32.294 < \chi_{0.95}^2(24) = 36.415
	\end{equation*}
	5. Decision: \\
	$ \chi^2 \notin W $, the null hypothesis cannot be rejected \\ \\
	12.3 \quad \textbf{B} \\
	test statistic: 
	\begin{equation*}
		\chi^2 = \frac{(n-1)s^2}{\sigma_0^2} \sim \chi^2(n-1) 
	\end{equation*} \\ \\
	12.4 \quad \textbf{D} \\
	rejection regions:
	\begin{equation*}
		W = \left\{ \bigg|\frac{\bar{x} - \mu_0}{s/\sqrt{n}}\bigg| > t_{1-\alpha/2}(n-1) \right\}
	\end{equation*} \\
	12.5 \quad \textbf{D} \\
	test statistic:
	\begin{equation*}
		t = \frac{\bar{x} - \mu}{s / \sqrt{n}}
	\end{equation*}
	and
	\begin{equation*}
		s = \sqrt{\frac{Q}{n-1}} \quad , \quad \mu = 4
	\end{equation*}
	substitute
	\begin{equation*}
		t = \sqrt{n(n-1)} \cdot \frac{\bar{x} - 4}{\sqrt{Q}}
	\end{equation*} \\
	12.6 \quad \textbf{B} \\
	1. Null hypothesis: $ \mu \geq 1000 $ \\
	Alternative hypothesis: $ \mu < 1000 $ \\
	2. Level of significance: $ \alpha = 0.05 $ \\
	3. Criterion: \\
	test statistic:
	\begin{equation*}
		u = \frac{\bar{x} - \mu_0}{\sigma / \sqrt{n}} \sim N(0, 1)
	\end{equation*}
	rejection regions:
	\begin{equation*}
		W = \{u < u_{\alpha}\}
	\end{equation*}
	4. Calculations:
	\begin{equation*}
		u = \frac{950 - 1000}{100 / \sqrt{25}} = -2.5 < u_{0.05} = -1.6
	\end{equation*}
	5. Decision: \\
	$ u \in W $, the null hypothesis must be rejected \\ \\
	12.7 \quad \textbf{B} \\
	1. Null hypothesis: $ \mu = 2.64 $ \\
	Alternative hypothesis: $ \mu \neq 2.64 $ \\
	2. Level of significance: $ \alpha = 0.01 $ \\
	3. Criterion: \\
	test statistic:
	\begin{equation*}
		u = \frac{\bar{x} - \mu_0}{\sigma / \sqrt{n}} \sim N(0, 1)
	\end{equation*}
	rejection regions:
	\begin{equation*}
		W = \{u < u_{\alpha / 2} \quad or \quad u > u_{1-\alpha / 2}\}
	\end{equation*}
	4. Calculations:
	\begin{equation*}
		u = \frac{2.61 - 2.64}{0.06 / \sqrt{36}} = -3 < u_{0.005} = -2.58
	\end{equation*}
	5. Decision: \\
	$ u \in W $, the null hypothesis must be rejected \\ \\
	12.8 \quad \textbf{C} \\
	I don't know ! \\ \\
	12.9 \quad \textbf{A} \\
	obviously \\ \\
	12.10 \quad \textbf{A} \\
	obviously \\ \\
	12.11 \quad \textbf{C} \\
	检验是否有显著增加,所以是单边假设 \\ \\
	12.12 \quad \textbf{C} \\
	1. 建立假设 
	\begin{equation*}
		H_0: \quad \mu \leq 0.8 \quad vs \quad H_1: \quad \mu > 0.8
	\end{equation*}
	2. 检验统计量
	\begin{equation*}
		t = \frac{\bar{x} - \mu_0}{s / \sqrt{n}} \sim t(n-1)
	\end{equation*}
	3. 显著性水平
	\begin{equation*}
		\alpha = 0.05
	\end{equation*}
	4. 拒绝域
	\begin{equation*}
		W = \{t > t_{1-\alpha}(n-1) = t_{0.95}(15)\} 
	\end{equation*}
	5. 做出判断
	因为
	\begin{equation*}
		t = \frac{0.92 - 0.8}{0.32 / \sqrt{16}} = 1.5 < t_{0.95}(15) = 1.753
	\end{equation*}
	即$ t \notin W $,所以不能够拒绝原假设,也就是说厂方断言正确
	\newpage
	\section{双正态总体分布的假设检验作业}
	\noindent
	13.1 \quad \textbf{D} \\
	calculate $ s^2 $
	\begin{equation*}
		\begin{split}
			s^2 &= \frac{1}{n-1} \sum\limits_{i=1}^{n}(x_i - \bar{x})^2 = \frac{1}{n-1} \left( \sum\limits_{i=1}^{n}x_i^2 + \sum\limits_{i=1}^{n}\bar{x}^2 - 2\bar{x}\sum\limits_{i=1}^{n}x_i \right) \\
			&= \frac{1}{n-1} \left( \sum\limits_{i=1}^{n}x_i^2 - n\bar{x} \right) = \frac{1}{n-1} \left( \sum\limits_{i=1}^{n}x_i^2 - \frac{1}{n} \left( \sum\limits_{i=1}^{n} x_i \right)^2 \right)
		\end{split}
	\end{equation*}
	calculate $ s_X $ and $ s_Y $
	\begin{equation*}
		\begin{split}
			s_X^2 &= \frac{1}{5} \left( 6978.93 - \frac{1}{6}\cdot (204.6)^2\right) = 0.414 \\
			&\\
			s_Y^2 &= \frac{1}{8} \left( 15280.173 - \frac{1}{9}\cdot (370.8)^2\right) = 0.402
		\end{split}
	\end{equation*}
	1. Null hypothesis: $ \sigma_X^2 = \sigma_Y^2 $ \\
	Alternative: $ \sigma_X^2 \neq \sigma_Y^2 $ \\
	2. Level of significance: $ \alpha = 0.05 $ \\
	3. Criterion: \\
	test statistic:
	\begin{equation*}
		F = \frac{s_X^2}{s_Y^2} \sim F(5, 8)
	\end{equation*}
	rejection regions:
	\begin{equation*}
		W = \{F < F_{\alpha / 2}(5, 8)\quad or \quad F > F_{1 - \alpha / 2}(5, 8)\}
	\end{equation*}
	Calculations: \\
	\begin{equation*}
		F = \frac{s_X^2}{s_Y^2} = 1.030 \in (0.207, 6.67)
	\end{equation*}
	Decision: \\
	$ F \notin W $, the null hypothesis cannot be rejected \\ \\ 
	13.2 \quad \textbf{A} \\
	obviously \\ \\
	13.3 \quad \textbf{B} \\
	1. Null hypothesis: $ \mu_1 \geq \mu_2 $ \\
	Alternative hypothesis: $ \mu_1 < \mu_2 $ \\
	2. Level of significance: $ \alpha = 0.01 $ \\
	3. Criterion: \\
	test statistic: 
	\begin{equation*}
		t = \frac{\bar{x} - \bar{y}}{\sqrt{\frac{s_1^2 + s_2^2}{n}}} \sim t(2n - 2)
	\end{equation*}
	rejection regions:
	\begin{equation*}
		W = \{t < t_{\alpha}(2n-2)\}
	\end{equation*}
	4. Calculations: \\
	\begin{equation*}
		t = -4.6468\textbf{ (by Excel) } < -2.5524
	\end{equation*}
	5. Decision: \\
	$ t \in W $, the null hypothesis must be rejected \\ \\
	13.4 \quad \textbf{C} \\
	1. Null hypothesis: $ \mu_1 = \mu_2 $ \\
	Althernative hypothesis: $ \mu_1 \neq \mu_2 $ \\
	2. Level of significance: $ \alpha = 0.05 $ \\
	3. Criterion: \\
	test statistic:
	\begin{equation*}
		u = \frac{\bar{x} - \bar{y}}{\sqrt{\frac{s_1^2}{m} + \frac{s_2^2}{n}}} \overset{\cdot}{\sim} N(0, 1)
	\end{equation*} 
	\textbf{note:} because of large sample \\
	rejection regions: \\
	\begin{equation*}
		W = \{|u| > u_{1-\alpha / 2}\}
	\end{equation*}
	4. Calculations: \\
	\begin{equation*}
		u = \frac{2805 - 2680}{\sqrt{\frac{120.96^2}{110} + \frac{105.53^2}{100}}} = 7.996 > 1.96
	\end{equation*}
	5. Decision: \\
	$ u \in W $, the Null hypothesis must be rejected \\ \\
	13.5 \quad \textbf{B} \\
	1. 建立假设
	\begin{equation*}
		H_0: \quad \sigma_1 \leq \sigma 2 \quad vs \quad H_1:\quad \sigma_1 > \sigma 2
	\end{equation*}
	2. 检验统计量
	\begin{equation*}
		F = \frac{s_1^2}{s_2^2} \sim F(9, 11)
	\end{equation*}
	3. 显著性水平
	\begin{equation*}
		\alpha = 0.005
	\end{equation*}
	4. 拒绝域
	\begin{equation*}
		W = \{F > F_{0.995}(9, 11)\}
	\end{equation*}
	5. 做出判断,因为
	\begin{equation*}
		F = \frac{50^2}{20^2} = 6.25 > 5.536
	\end{equation*}
	即$ F \in W $,所以拒绝原假设,也就是说新生产工艺下灯管寿命的稳定性显著提高
	\newpage
	\section{非参数假设检验}
	\noindent
	14.1 \quad \textbf{A} \\
	obviously \\ \\
	14.2 \quad \textbf{D}
	\begin{table}[!htbp]
		\centering
		\begin{tabular}{c|ccc}
			\hline
			公司 & A & B & others \\ \hline
			理论频率 & 45\% & 40\% & 15\% \\ \hline
			实测频数 & 102 & 82 & 16 \\ \hline
		\end{tabular}
	\end{table} \\
	1. 提出假设
	\begin{equation*}
		\begin{split}
			H_0:& \quad \pi_A = 0.45 \wedge \pi_B = 0.40 \wedge \pi_O = 0.15 \\
			H_1:& \quad \pi_A \neq 0.45 \vee \pi_B \neq 0.40 \vee \pi_O \neq 0.15
		\end{split}
	\end{equation*}
	\textbf{注:}仅记A公司的市场占有率为$ \pi_A $ \\
	2. 显著性水平
	\begin{equation*}
		\alpha = 0.05
	\end{equation*}
	3. 标准 \\
	检验统计量
	\begin{equation*}
		\chi^2 = \sum\limits_{i=1}^{k} \frac{(f_i - np_i)^2}{np_i} \sim \chi^2(2)
	\end{equation*}
	拒绝域 \\
	\begin{equation*}
		W = \{\chi^2 > \chi_{0.95}(2)\}
	\end{equation*}
	4. 计算
	\begin{equation*}
		\chi^2 = \sum\limits_{i=1}^{3} \frac{(f_i - np_i)^2}{np_i} = 8.183 > 5.991
	\end{equation*}
	5. 决定 \\
	$ \chi^2 \in W $,拒绝原假设,广告战后各公司市场占有率有显著变化  \\ \\
	14.3 \quad \textbf{B} \\
	1. 提出假设
	\begin{equation*}
		H_0: \quad \text{骰子是均匀的} \quad vs \quad H_1: \quad \text{骰子是不均匀的}
	\end{equation*}
	2. 显著性水平
	\begin{equation*}
		\alpha = 0.05
	\end{equation*}
	3. 标准 \\
	检验统计量
	\begin{equation*}
		\chi^2 = \sum\limits_{i=1}^{k} \frac{(f_i - np_i)^2}{np_i} \sim \chi^2(5)
	\end{equation*}
	拒绝域
	\begin{equation*}
		W = \{\chi^2 > \chi_{0.95}(5)\}
	\end{equation*}
	4. 计算
	\begin{equation*}
		\chi^2 = \sum\limits_{i=1}^{6} \frac{(f_i - np_i)^2}{np_i} = 18.7 > 11.07
	\end{equation*}
	\textbf{注:}$ np_i = 120 \times 1/6 = 20 $\\
	5. 决定 \\
	$ \chi^2 \in W $,拒绝原假设,骰子是不均匀的 \\
	\begin{figure}[!htbp]
		\centering
		\includegraphics{abab.jpg}
	\end{figure}
	\newpage
	\section{附录}
	\noindent
	\textbf{\large I. $ \Gamma $函数和B函数}(P102 - P105) \\
	\begin{equation*}
		\begin{split}
			\Gamma(\alpha) &= \int_{0}^{+\infty} x^{\alpha - 1}e^{-x}dx \\
			& \\
			B(a, b) &= \int_{0}^{1} x^{a-1}(1-x)^{b-1}dx = \frac{\Gamma(a)\Gamma(b)}{\Gamma(a+b)}
		\end{split}
	\end{equation*} \\
	\textbf{\large II. 区间估计}
	\begin{table}[htbp!]
		\centering
		\renewcommand\arraystretch{1.8}
		\begin{tabular}{|c|c|c|c|c|c|c|}
			\hline
			\multirow{2}{*}{点估计} & \multirow{2}{*}{总体分布} & \multirow{2}{*}{样本容量} & \multicolumn{2}{|c|}{$\sigma$ 未知} & \multicolumn{2}{|c|}{$\sigma$已知} \\
			\cline{4-7}
			& & & 统计量 & 区间 & 统计量 & 区间 \\ \hline
			\multirow{4}{*}{总体均值$\mu$} & \multirow{2}{*}{正态分布} & {小样本 ($ n < 30 $)} & $ t = \frac{\bar{x} - \mu}{s/\sqrt{n}} $ & $ \bar{x} \pm t_{\alpha/2} \cdot \frac{s}{\sqrt{n}} $ & \multirow{3}{*}{$ z=\frac{\bar{x}-\mu}{\sigma / \sqrt{n}} $} & \multirow{3}{*}{$ \bar{x} \pm z_{\alpha / 2} \cdot \frac{\sigma}{\sqrt{n}} $} \\
			\cline{3-5}
			& & 大样本($ n \geq 30 $) & \multirow{2}{*}{$ z= \frac{\bar{x}-\mu}{s/\sqrt{n}} $} & \multirow{2}{*}{$\bar{x} \pm z_{\alpha/2} \cdot \frac{s}{\sqrt{n}}$} & & \\
			\cline{2-3}
			 & \multirow{2}{*}{非正态分布} & 大样本($ n \geq 30 $) & & & & \\
			\cline{3-7}
			 & & 小样本($ n < 30 $) & \multicolumn{4}{|c|}{具体分布未知} \\
			\cline{1-7} 
			\multirow{2}{*}{总体方差$ \sigma^2 $} & \multirow{2}{*}{正态分布} & \multirow{2}{*}{无要求} & $ \chi^2 = \frac{(n-1)s^2}{\sigma^2} $ & \multicolumn{3}{|c|}{$ \left[ \frac{(n-1)s^2}{\chi_{\alpha / 2}^2(n-1)} , \frac{(n-1)s^2}{\chi_{1-\alpha / 2}^2(n-1)}\right] $\quad ($ \mu $未知)} \\
			\cline{4-7}
			& & & $ z = \frac{\bar{x} - \mu}{\sigma / \sqrt{n}} $ & \multicolumn{3}{|c|}{$ \sigma > \frac{(\bar{x} - \mu)\sqrt{n}}{z_{\alpha / 2}} $ \quad ($\mu$已知)} \\
			\cline{1-7}
		\end{tabular}
	\end{table} \\
	\textbf{\large III. F、t、u分布分位数}
	\begin{equation*}
		F_{\alpha} (n, m) = \frac{1}{F_{1-\alpha}(m, n)}
	\end{equation*}
	\begin{equation*}
		\begin{split}
			u_{1-\alpha} &= -u_{\alpha} \\
			t_{1-\alpha} &= -t_{\alpha} 
		\end{split}
	\end{equation*}
	\textbf{\large IV. 勘误}
	\begin{tcolorbox}
		[
		colframe=red!70,
		coltitle=red!10!white, 
		fonttitle=\bfseries, 
		adjusted title=P6 \quad 2.5
		]
		\begin{equation*}
			\Phi\left(\frac{1000 - 10n}{\sqrt{n}}\right) \geq 0.99
		\end{equation*}
		查表得
		\begin{equation*}
			\Phi(2.33) = 0.9901
		\end{equation*}
		那么
		\begin{equation*}
			\frac{1000 - 10n}{\sqrt{n}} \geq 2.33 \quad \Rightarrow \quad n = 98
		\end{equation*}
	\end{tcolorbox}
\end{document}